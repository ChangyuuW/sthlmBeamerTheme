% 	Name		:: 	sthlm Beamer Theme  HEAVILY based on the hsrmbeamer theme (Benjamin Weiss) 
%	Author		:: 	Mark Hendry Olson (mark@hendryolson.com)
%	Created		::	2013-07-31
%	Updated		::	April 19, 2015 at 16:46:26  
%	Version		:: 	0.5
%	Email		:: 	mark@hendryolson.com
%	Website		:: 	http://hendryolson.com
%
% 	License		:: 	This file may be distributed and/or modified under the
%                  	GNU Public License.
%
%	Description	::	This presentation is a demonstration of the sthlm beamer 
%					theme, which is HEAVILY based on the HSRM beamer theme created by Benjamin Weiss 
%					(benjamin.weiss@student.hs-rm.de), which can be found on GitHub
%					<https://github.com/hsrmbeamertheme/hsrmbeamertheme>.	


%-=-=-=-=-=-=-=-=-=-=-=-=-=-=-=-=-=-=-=-=-=-=-=-=
%
%        LOADING DOCUMENT
%
%-=-=-=-=-=-=-=-=-=-=-=-=-=-=-=-=-=-=-=-=-=-=-=-=

\documentclass[compress,PxFont]{beamer}
%\documentclass[compress]{beamer}
\usetheme{sthlm}


%-=-=-=-=-=-=-=-=-=-=-=-=-=-=-=-=-=-=-=-=-=-=-=-=
%        LOADING PACKAGES
%-=-=-=-=-=-=-=-=-=-=-=-=-=-=-=-=-=-=-=-=-=-=-=-=
\usepackage[utf8]{inputenc}
\usepackage{
booktabs,
datetime,
dtklogos,
pgfplots,
ragged2e,
tabularx,
wasysym
}

\pgfplotsset{compat=1.8}

%-=-=-=-=-=-=-=-=-=-=-=-=-=-=-=-=-=-=-=-=-=-=-=-=
%        BEAMER OPTIONS
%-=-=-=-=-=-=-=-=-=-=-=-=-=-=-=-=-=-=-=-=-=-=-=-=

\setbeameroption{show notes}

%-=-=-=-=-=-=-=-=-=-=-=-=-=-=-=-=-=-=-=-=-=-=-=-=
%        LOADING TIKZ LIBRARIES
%-=-=-=-=-=-=-=-=-=-=-=-=-=-=-=-=-=-=-=-=-=-=-=-=

\usetikzlibrary{
backgrounds,
mindmap
}

%-=-=-=-=-=-=-=-=-=-=-=-=-=-=-=-=-=-=-=-=-=-=-=-=
%
%	PRESENTATION INFORMATION
%
%-=-=-=-=-=-=-=-=-=-=-=-=-=-=-=-=-=-=-=-=-=-=-=-=

\title{Stockholm Beamer Theme}
\subtitle{sthlm version 0.6 is based on the hsrm theme}
\date{\small{\jobname}}
\author{\texttt{HendryOlson.com}}
\institute{Made in \textit{Sweden}}

\hypersetup{
pdfauthor = {Mark H. Olson: mark@hendryolson.com},      
pdfsubject = {},
pdfkeywords = {},  
pdfmoddate= {D:\pdfdate},          
pdfcreator = {}
}

\begin{document}

%-=-=-=-=-=-=-=-=-=-=-=-=-=-=-=-=-=-=-=-=-=-=-=-=
%
%	TITLE PAGE
%
%-=-=-=-=-=-=-=-=-=-=-=-=-=-=-=-=-=-=-=-=-=-=-=-=

\maketitle

%\begin{frame}[plain]
%	\titlepage
%\end{frame}

%-=-=-=-=-=-=-=-=-=-=-=-=-=-=-=-=-=-=-=-=-=-=-=-=
%
%	TABLE OF CONTENTS: OVERVIEW
%
%-=-=-=-=-=-=-=-=-=-=-=-=-=-=-=-=-=-=-=-=-=-=-=-=

\section*{Overview}
\begin{frame}{Overview}
% For longer presentations use hideallsubsections option
\tableofcontents[hideallsubsections]
\end{frame}

%-=-=-=-=-=-=-=-=-=-=-=-=-=-=-=-=-=-=-=-=-=-=-=-=
%
%	SECTION: BACKGROUND
%
%-=-=-=-=-=-=-=-=-=-=-=-=-=-=-=-=-=-=-=-=-=-=-=-=

\section{Background}

%-=-=-=-=-=-=-=-=-=-=-=-=-=-=-=-=-=-=-=-=-=-=-=-=
%	FRAME: What is Beamer?
%-=-=-=-=-=-=-=-=-=-=-=-=-=-=-=-=-=-=-=-=-=-=-=-=

\begin{frame}{What is Beamer?}

\centerline {Beamer is a \LaTeX\ class for creating beautiful presentations.}  

\begin{block}{The sthlm Beamer Theme:}
\begin{itemize}
	\item written using utf8 encoding
	\item should compile using pdfLaTeX
\end{itemize}
\end{block}

\end{frame}

%-=-=-=-=-=-=-=-=-=-=-=-=-=-=-=-=-=-=-=-=-=-=-=-=
%
%	SECTION: UPDATESS
%
%-=-=-=-=-=-=-=-=-=-=-=-=-=-=-=-=-=-=-=-=-=-=-=-=

\section{Updates}

%-=-=-=-=-=-=-=-=-=-=-=-=-=-=-=-=-=-=-=-=-=-=-=-=
%	FRAME: 
%-=-=-=-=-=-=-=-=-=-=-=-=-=-=-=-=-=-=-=-=-=-=-=-=

\begin{frame}[c]{Version 0.6}
\alert{NeW} in version 0.6

\begin{itemize}
	\item \texttt{beamerthemesthlm.sty}
	\begin{itemize}
		\item Removed log from headeer
		\item Removed vertically aligned columns support
		\item Removed listings package support
		\item Added PxFont option to use the fonts:\footnote{This presentation is using the NewPX fonts}
		\begin{itemize}
			\item newpxtext
			\item newpxmath 
		\end{itemize}
	\end{itemize}
\end{itemize}
\end{frame}


%-=-=-=-=-=-=-=-=-=-=-=-=-=-=-=-=-=-=-=-=-=-=-=-=
%
%	SECTION: STRUCTURE
%
%-=-=-=-=-=-=-=-=-=-=-=-=-=-=-=-=-=-=-=-=-=-=-=-=
\section{Structure}

%-=-=-=-=-=-=-=-=-=-=-=-=-=-=-=-=-=-=-=-=-=-=-=-=
%	FRAME: Primary Presentation Colors
%-=-=-=-=-=-=-=-=-=-=-=-=-=-=-=-=-=-=-=-=-=-=-=-=

\begin{frame}{Primary Presentation Colors}

\begin{columns}
\begin{column}{.48\linewidth}
	
\setbeamercolor{sthlmLightRed}{fg=sthlmLightRed,bg=white}
\begin{beamercolorbox}[wd=\linewidth,ht=2ex,dp=0.7ex]{sthlmLightRed}
	\texttt{sthlmLightRed}
\end{beamercolorbox}

\setbeamercolor{sthlmGreen}{fg=sthlmGreen,bg=white}
\begin{beamercolorbox}[wd=\linewidth,ht=2ex,dp=0.7ex]{sthlmGreen}
	\texttt{sthlmGreen}
\end{beamercolorbox}

\setbeamercolor{sthlmDarkBlue}{fg=sthlmDarkBlue,bg=white}
\begin{beamercolorbox}[wd=\linewidth,ht=2ex,dp=0.7ex]{sthlmDarkBlue}
	\texttt{sthlmDarkBlue}
\end{beamercolorbox}

\setbeamercolor{sthlmDarkGrey}{fg=sthlmDarkGrey,bg=white}
\begin{beamercolorbox}[wd=\linewidth,ht=2ex,dp=0.7ex]{sthlmDarkGrey}
	\texttt{sthlmDarkGrey}
\end{beamercolorbox}

\setbeamercolor{sthlmLightGrey}{fg=sthlmLightGrey,bg=white}
\begin{beamercolorbox}[wd=\linewidth,ht=2ex,dp=0.7ex]{sthlmLightGrey}
	\texttt{sthlmLightGrey}
\end{beamercolorbox}

\end{column}

\begin{column}{.48\linewidth}
\setbeamercolor{sthlmLightRed}{bg=sthlmLightRed,fg=white}
\begin{beamercolorbox}[wd=\linewidth,ht=2ex,dp=0.7ex]{sthlmLightRed}
	\texttt{sthlmLightRed}
\end{beamercolorbox}

\setbeamercolor{sthlmGreen}{bg=sthlmGreen,fg=black}
\begin{beamercolorbox}[wd=\linewidth,ht=2ex,dp=0.7ex]{sthlmGreen}
	\texttt{sthlmGreen}
\end{beamercolorbox}

\setbeamercolor{sthlmDarkBlue}{bg=sthlmDarkBlue,fg=white}
\begin{beamercolorbox}[wd=\linewidth,ht=2ex,dp=0.7ex]{sthlmDarkBlue}
	\texttt{sthlmDarkBlue}
\end{beamercolorbox}

\setbeamercolor{sthlmDarkGrey}{bg=sthlmDarkGrey,fg=white}
\begin{beamercolorbox}[wd=\linewidth,ht=2ex,dp=0.7ex]{sthlmDarkGrey}
	\texttt{sthlmDarkGrey}
\end{beamercolorbox}

\setbeamercolor{sthlmLightGrey}{bg=sthlmLightGrey,fg=black}
\begin{beamercolorbox}[wd=\linewidth,ht=2ex,dp=0.7ex]{sthlmLightGrey}
	\texttt{sthlmLightGrey}
\end{beamercolorbox}
\end{column}
\end{columns}
\end{frame}

%-=-=-=-=-=-=-=-=-=-=-=-=-=-=-=-=-=-=-=-=-=-=-=-=
%	FRAME: Secondary Presentation Colors
%-=-=-=-=-=-=-=-=-=-=-=-=-=-=-=-=-=-=-=-=-=-=-=-=

\begin{frame}{Secondary Presentation Colors}
\begin{columns}
\begin{column}{.48\linewidth}
	
\setbeamercolor{sthlmRed}{fg=sthlmRed,bg=white}
\begin{beamercolorbox}[wd=\linewidth,ht=2ex,dp=0.7ex]{sthlmRed}
	\texttt{sthlmRed}
\end{beamercolorbox}

\setbeamercolor{sthlmYellow}{fg=sthlmYellow,bg=white}
\begin{beamercolorbox}[wd=\linewidth,ht=2ex,dp=0.7ex]{sthlmYellow}
	\texttt{sthlmYellow}
\end{beamercolorbox}

\setbeamercolor{sthlmLightYellow}{fg=sthlmLightYellow,bg=white}
\begin{beamercolorbox}[wd=\linewidth,ht=2ex,dp=0.7ex]{sthlmLightYellow}
	\texttt{sthlmLightYellow}
\end{beamercolorbox}

\setbeamercolor{sthlmLightBlue}{fg=sthlmLightBlue,bg=white}
\begin{beamercolorbox}[wd=\linewidth,ht=2ex,dp=0.7ex]{sthlmLightBlue}
	\texttt{sthlmLightBlue}
\end{beamercolorbox}

\setbeamercolor{sthlmBlue}{fg=sthlmBlue,bg=white}
\begin{beamercolorbox}[wd=\linewidth,ht=2ex,dp=0.7ex]{sthlmBlue}
	\texttt{sthlmBlue}
\end{beamercolorbox}

\setbeamercolor{sthlmPurple}{fg=sthlmPurple,bg=white}
\begin{beamercolorbox}[wd=\linewidth,ht=2ex,dp=0.7ex]{sthlmPurple}
	\texttt{sthlmPurple}
\end{beamercolorbox}

\setbeamercolor{sthlmGrey}{fg=sthlmGrey,bg=white}
\begin{beamercolorbox}[wd=\linewidth,ht=2ex,dp=0.7ex]{sthlmGrey}
	\texttt{sthlmGrey}
\end{beamercolorbox}

\end{column}

\begin{column}{.48\linewidth}
\setbeamercolor{sthlmRed}{bg=sthlmRed,fg=black}
\begin{beamercolorbox}[wd=\linewidth,ht=2ex,dp=0.7ex]{sthlmRed}
	\texttt{sthlmRed}
\end{beamercolorbox}

\setbeamercolor{sthlmYellow}{bg=sthlmYellow,fg=black}
\begin{beamercolorbox}[wd=\linewidth,ht=2ex,dp=0.7ex]{sthlmYellow}
	\texttt{sthlmYellow}
\end{beamercolorbox}

\setbeamercolor{sthlmLightYellow}{bg=sthlmLightYellow,fg=black}
\begin{beamercolorbox}[wd=\linewidth,ht=2ex,dp=0.7ex]{sthlmLightYellow}
	\texttt{sthlmLightYellow}
\end{beamercolorbox}

\setbeamercolor{sthlmLightBlue}{bg=sthlmLightBlue,fg=black}
\begin{beamercolorbox}[wd=\linewidth,ht=2ex,dp=0.7ex]{sthlmLightBlue}
	\texttt{sthlmLightBlue}
\end{beamercolorbox}

\setbeamercolor{sthlmBlue}{bg=sthlmBlue,fg=white}
\begin{beamercolorbox}[wd=\linewidth,ht=2ex,dp=0.7ex]{sthlmBlue}
	\texttt{sthlmBlue}
\end{beamercolorbox}

\setbeamercolor{sthlmPurple}{bg=sthlmPurple,fg=white}
\begin{beamercolorbox}[wd=\linewidth,ht=2ex,dp=0.7ex]{sthlmPurple}
	\texttt{sthlmPurple}
\end{beamercolorbox}

\setbeamercolor{sthlmGrey}{bg=sthlmGrey,fg=white}
\begin{beamercolorbox}[wd=\linewidth,ht=2ex,dp=0.7ex]{sthlmGrey}
	\texttt{sthlmGrey}
\end{beamercolorbox}

\end{column}
\end{columns}
\end{frame}

%-=-=-=-=-=-=-=-=-=-=-=-=-=-=-=-=-=-=-=-=-=-=-=-=
%	FRAME: Theme Package Requirements
%-=-=-=-=-=-=-=-=-=-=-=-=-=-=-=-=-=-=-=-=-=-=-=-=

\begin{frame}[containsverbatim]{Theme Package Requirements}

This theme requires that:

\begin{enumerate}
	\item it is compiled using pdflatex 
	\item the following packages are installed
	\begin{enumerate}
		\item \verb|{beamer}|
		\item \verb|{eso-pic}|
		\item \verb|[utf8]{inputenc}|
		\item \verb|{pgf}|
		\item \verb|{verbatim}|
		\item \verb|{xcolor}|
	\end{enumerate}
\end{enumerate}

\end{frame}

%-=-=-=-=-=-=-=-=-=-=-=-=-=-=-=-=-=-=-=-=-=-=-=-=
%	FRAME: Theme Fonts
%-=-=-=-=-=-=-=-=-=-=-=-=-=-=-=-=-=-=-=-=-=-=-=-=

\begin{frame}[containsverbatim]{Theme Fonts}

To use the newpx fonts option you need the following packages:

\begin{itemize}
	\item \verb|{newpxtext}|
	\item \verb|{newpxmath}|
	\item \verb|[T1]{fontenc}|
\end{itemize}

To enable the \verb|newpxtext| and \verb|newpxmath| fonts include the \alert{PxFont} class option.\\

\begin{itemize}
	\item \verb|\documentclass[compress,PxFont]{beamer}|
\end{itemize}

\end{frame}

%-=-=-=-=-=-=-=-=-=-=-=-=-=-=-=-=-=-=-=-=-=-=-=-=
%	FRAME: Theme Options
%-=-=-=-=-=-=-=-=-=-=-=-=-=-=-=-=-=-=-=-=-=-=-=-=

\begin{frame}{Theme Options}
This theme comes with some options to change it's appearance.
\begin{table}[]
	\begin{tabularx}{\linewidth}{l>{\raggedright}X}
		\toprule
		\textbf{Option}			& \textbf{Description} \tabularnewline
		\midrule
		\texttt{nosectionpages} & Section pages will be supressed.\tabularnewline
		\texttt{PxFont} & newpxtext and newpxtext will be used.\tabularnewline
		\bottomrule
	\end{tabularx}
	\label{tab:options}
\end{table}
\end{frame}

%-=-=-=-=-=-=-=-=-=-=-=-=-=-=-=-=-=-=-=-=-=-=-=-=
%	FRAME: Blocks
%-=-=-=-=-=-=-=-=-=-=-=-=-=-=-=-=-=-=-=-=-=-=-=-=

\begin{frame}[containsverbatim]{Blocks}
The default Beamer Box
\begin{block}{Block Title Here}
	\begin{itemize}
		\item point 1
		\item point 2
	\end{itemize}
\end{block}
\begin{verbatim}
\begin{block}{Block Title Here}
    \begin{itemize}
        \item point 1
        \item point 2
    \end{itemize}
\end{block}
\end{verbatim}
\end{frame}

%-=-=-=-=-=-=-=-=-=-=-=-=-=-=-=-=-=-=-=-=-=-=-=-=
%	FRAME: Additional Blocks
%-=-=-=-=-=-=-=-=-=-=-=-=-=-=-=-=-=-=-=-=-=-=-=-=

\begin{frame}[containsverbatim]{Additional Blocks}
\begin{alertblock}{Alert Block}
	Highlight important information.
\end{alertblock}
\begin{verbatim}
\begin{alertblock}{Alert Block}
    Highlight important information.
\end{alertblock}
\end{verbatim}

\end{frame}

%-=-=-=-=-=-=-=-=-=-=-=-=-=-=-=-=-=-=-=-=-=-=-=-=
%	FRAME: Additional Blocks
%-=-=-=-=-=-=-=-=-=-=-=-=-=-=-=-=-=-=-=-=-=-=-=-=

\begin{frame}[containsverbatim]{Additional Blocks}

\begin{exampleblock}{Example Block}
	Examples can be good.
\end{exampleblock}
\begin{verbatim}
\begin{exampleblock}{Example Block}
    Examples can be good
\end{exampleblock}
\end{verbatim}
\end{frame}

%-=-=-=-=-=-=-=-=-=-=-=-=-=-=-=-=-=-=-=-=-=-=-=-=
%	FRAME: Blocks
%-=-=-=-=-=-=-=-=-=-=-=-=-=-=-=-=-=-=-=-=-=-=-=-=

\begin{frame}[containsverbatim]{Custom Blocks}
\begingroup
\setbeamercolor{block title}{fg=black, bg=sthlmGreen}
\setbeamercolor{block body}{bg=sthlmLightGrey}
\begin{block}{Green customization}
	Using the theme colors to generate colored blocks.
\end{block}
\endgroup
\begin{verbatim}
\begingroup
\setbeamercolor{block title}{fg=black, bg=sthlmGreen}
\setbeamercolor{block body}{bg=sthlmLightGrey}
\begin{block}{Custom Blocks}
    Using the theme colors to generate colored blocks.
\end{block}
\endgroup
\end{verbatim}
\end{frame}

%-=-=-=-=-=-=-=-=-=-=-=-=-=-=-=-=-=-=-=-=-=-=-=-=
%
%	SECTION: ADDITIONAL FEATURES
%
%-=-=-=-=-=-=-=-=-=-=-=-=-=-=-=-=-=-=-=-=-=-=-=-=
\section{Features}

%-=-=-=-=-=-=-=-=-=-=-=-=-=-=-=-=-=-=-=-=-=-=-=-=
%	FRAME: Images
%-=-=-=-=-=-=-=-=-=-=-=-=-=-=-=-=-=-=-=-=-=-=-=-=

\begin{frame}{Images with Copyright}
	\begin{figure}
		\centering
		\includegraphicscopyright[width=\linewidth]{photo.jpg}{Copyright by \href{http://netzlemming.deviantart.com/}{Netzlemming}, \href{http://creativecommons.org/licenses/by-nc/3.0/}{CC BY-NC 3.0 License}}
	\end{figure}
\end{frame}

%-=-=-=-=-=-=-=-=-=-=-=-=-=-=-=-=-=-=-=-=-=-=-=-=
%	FRAME: Tables
%-=-=-=-=-=-=-=-=-=-=-=-=-=-=-=-=-=-=-=-=-=-=-=-=

\begin{frame}{Tables}
\begin{table}[]
	\caption{Selection of window function and their properties}
	\begin{tabular}[]{lrrr}
		\toprule
		\textbf{Window}			& \multicolumn{1}{c}{\textbf{First side lobe}}	
		                    & \multicolumn{1}{c}{\textbf{3\,dB bandwidth}}
		                    & \multicolumn{1}{c}{\textbf{Roll-off}} \\
		\midrule
		Rectangular				& 13.2\,dB	& 0.886\,Hz/bin	& 6\,dB/oct		\\[0.25em]
		Triangular				& 26.4\,dB	& 1.276\,Hz/bin	& 12\,dB/oct	\\[0.25em]
		Hann					& 31.0\,dB	& 1.442\,Hz/bin	& 18\,dB/oct	\\[0.25em]
		Hamming					& 41.0\,dB	& 1.300\,Hz/bin	& 6\,dB/oct		\\
		\bottomrule
	\end{tabular}
	\label{tab:WindowFunctions}
\end{table}
\end{frame}

%-=-=-=-=-=-=-=-=-=-=-=-=-=-=-=-=-=-=-=-=-=-=-=-=
%	FRAME: Formulas
%-=-=-=-=-=-=-=-=-=-=-=-=-=-=-=-=-=-=-=-=-=-=-=-=

\begin{frame}{Formulas}
\begin{block}{Fourier Integral}
\[
F(\textrm{j}\omega) = \displaystyle \int \limits_{-\infty}^{\infty} \! f(t)\cdot\textrm{e}^{-\textrm{j}\omega t}  \, \mathrm{d} x
\]
\end{block}
\end{frame}

%-=-=-=-=-=-=-=-=-=-=-=-=-=-=-=-=-=-=-=-=-=-=-=-=
%	FRAME: PGFPlots
%-=-=-=-=-=-=-=-=-=-=-=-=-=-=-=-=-=-=-=-=-=-=-=-=

\begin{frame}{PGFPlots Bar Plot Example}
	\begin{figure}[h]
		\centering
		\begin{tikzpicture}
		\begin{axis}[
		    ybar,
		    enlarge x limits=0.15,
		    legend style={at={(-.5,0.5)},
		      anchor=north,legend columns=1},
		    ylabel={\% students},
		    symbolic x coords={A, B, C, D, E, F, DNF},
		    xtick=data,
		     bar width=2mm,
		     width=0.7\textwidth
		    ]
		    \legend{2012, 2013};
		    % Spring 2012 results
			\addplot[fill=sthlmLightGrey]  coordinates {(A,0) (B,0) (C,3.85) (D,23.07) (E,43.31) (F,30.77) (DNF,0.00)};
			% Spring 2013 results
			\addplot[fill=sthlmDarkBlue]   coordinates {(A,0) (B,3.70) (C,22.22) (D,22.22) (E,40.74) (F,3.70) (DNF,7.41)};
		\end{axis}
		\end{tikzpicture}
		\caption{Consistent improvement over the last year}
		\end{figure}
\end{frame}


%-=-=-=-=-=-=-=-=-=-=-=-=-=-=-=-=-=-=-=-=-=-=-=-=
%	FRAME: PGFPlots
%-=-=-=-=-=-=-=-=-=-=-=-=-=-=-=-=-=-=-=-=-=-=-=-=

\begin{frame}{PGFPlots 3-D Example}
	\begin{figure}[h]
		\centering
		\begin{tikzpicture}
			\begin{axis}[
			  axis lines=left,
			  axis on top,
			  xlabel={$x$}, ylabel={$y$}, zlabel={$z$},
			  domain=0:1,
			  y domain=0:2*pi,
			  xmin=-1.5, xmax=1.5,
			  ymin=-1.5, ymax=1.5, zmin=0.0,
			  mesh/interior colormap=
			  	{blueblack}{color=(black) color=(sthlmDarkBlue)},
			  colormap/blackwhite, 
			  samples=10,
			  samples y=40,
			  z buffer=sort,
			 ]
			  \addplot3[surf] 
			  	({x*cos(deg(y))},{x*sin(deg(y))},{x});
			\end{axis}
\end{tikzpicture}
		\caption{Easily add beautiful plots}
		\end{figure}
\end{frame}

\begin{frame}{Mindmap made with TikZ}
\centering
\begin{tikzpicture}[scale=0.88]
	\path[mindmap,concept color=sthlmPurple,text=white]
	node[concept] {\textcolor{white}\TeX}
	[clockwise from=-30]
	child[concept color=sthlmBlue,text=white] { node[concept] {\textcolor{white}{\XeTeX}} }
	child[concept color=sthlmLightBlue,text=white] { node[concept] {\ConTeXt} }
	child[concept color=sthlmBlue,text=white] { node[concept] {\LaTeX} };
\end{tikzpicture}
\end{frame}

%-=-=-=-=-=-=-=-=-=-=-=-=-=-=-=-=-=-=-=-=-=-=-=-=
%
%	SECTION: TUTORIAL
%
%-=-=-=-=-=-=-=-=-=-=-=-=-=-=-=-=-=-=-=-=-=-=-=-=

\section{Tutorial}

%-=-=-=-=-=-=-=-=-=-=-=-=-=-=-=-=-=-=-=-=-=-=-=-=
%	FRAME: Presentation Structure
%-=-=-=-=-=-=-=-=-=-=-=-=-=-=-=-=-=-=-=-=-=-=-=-=

\begin{frame}[containsverbatim]{Presentation Structure}

A section page will be generated and the section name included in the presentation header for each section of the presentation with the current section being emphasized.  If you include subsections in your presentation, then a small block will appear under the section name in the header for each frame.  Once a frame has been viewed it will turn green.

It's worth noting that a frame can make up multiple slides.

\begin{verbatim}
\section{Main Section}
\subsection{Main Subsection}
\begin{frame}
\frametitle{Presentation Stucture}
% Frame Contents Here
\end{frame}
\end{verbatim}
\end{frame}

%-=-=-=-=-=-=-=-=-=-=-=-=-=-=-=-=-=-=-=-=-=-=-=-=
%	FRAME: Table of Contents
%-=-=-=-=-=-=-=-=-=-=-=-=-=-=-=-=-=-=-=-=-=-=-=-=

\begin{frame}[containsverbatim]{Table of Contents}
Include a listing of the presentation's sections 
\begin{verbatim}
\maketitle
\end{verbatim}
For those longer presentations - keep the table of contents compact.
\begin{verbatim}
\begin{frame}{Overview}
    \tableofcontents[hideallsubsections]
\end{frame}
\end{verbatim}
\end{frame}

%-=-=-=-=-=-=-=-=-=-=-=-=-=-=-=-=-=-=-=-=-=-=-=-=
%	FRAME: Quotations
%-=-=-=-=-=-=-=-=-=-=-=-=-=-=-=-=-=-=-=-=-=-=-=-=

\begin{frame}[containsverbatim]{Quotations}
At any time you can highlight text by using the definition:
\begin{itemize}
	\item \alert{This is super important!}
\end{itemize}

\end{frame}

%-=-=-=-=-=-=-=-=-=-=-=-=-=-=-=-=-=-=-=-=-=-=-=-=
%	FRAME: Notes
%-=-=-=-=-=-=-=-=-=-=-=-=-=-=-=-=-=-=-=-=-=-=-=-=

\begin{frame}{Notes}
    
There are a couple different ways in which you can present your presentation.

\begin{itemize}
\item Splitshow (Mac OS X)\\\url{https://code.google.com/p/splitshow/}
\item pdf-presenter (Windows)\\\url{https://code.google.com/p/pdf-presenter/}
\end{itemize}
\end{frame}

\note{
Let me include a note for this particular slide.
	
Lorem ipsum dolor sit amet, consectetur adipisicing elit, sed do eiusmod tempor incididunt ut labore et dolore magna aliqua. Ut enim ad minim veniam, quis nostrud exercitation ullamco laboris nisi ut aliquip ex ea commodo consequat. Duis aute irure dolor in reprehenderit in voluptate velit esse cillum dolore eu fugiat nulla pariatur. Excepteur sint occaecat cupidatat non proident, sunt in culpa qui officia deserunt mollit anim id est laborum.

}

%-=-=-=-=-=-=-=-=-=-=-=-=-=-=-=-=-=-=-=-=-=-=-=-=
%	FRAME: Multiple Columns
%-=-=-=-=-=-=-=-=-=-=-=-=-=-=-=-=-=-=-=-=-=-=-=-=

\begin{frame}{Multiple Columns}
\begin{columns}
\begin{column}{.48\linewidth}
		Lorem ipsum dolor sit amet, consectetur adipisicing elit, sed do eiusmod
		tempor incididunt ut labore et dolore magna aliqua. Ut enim ad minim veniam,
		quis nostrud exercitation ullamco laboris nisi ut aliquip ex ea commodo
		consequat. Duis aute irure dolor in reprehenderit in voluptate velit esse
		cillum dolore eu fugiat nulla pariatur. 
\end{column}
\begin{column}{.48\linewidth}
		\begin{itemize}
        	\item Point 1
        	\item Point 2
		\end{itemize}
	\end{column}
	\end{columns}
\end{frame}

\begin{frame}{References}
	\begin{thebibliography}{10}
    
	\beamertemplatebookbibitems
	\bibitem{Oppenheim2009}
	Alan V. Oppenheim
	\newblock Discrete - Time Signal Processing
	\newblock Prentice Hall Press, 2009

	\beamertemplatearticlebibitems
	\bibitem{EBU2011}
	European Broadcasting Union
	\newblock Specification of the Broadcast Wave Format (BWF)
	\newblock 2011
  \end{thebibliography}
\end{frame}

%-=-=-=-=-=-=-=-=-=-=-=-=-=-=-=-=-=-=-=-=-=-=-=-=
%
%	SECTION: Conclusion
%
%-=-=-=-=-=-=-=-=-=-=-=-=-=-=-=-=-=-=-=-=-=-=-=-=

\begin{frame}{About}
	
This sthlm beamer theme is free software: you can redistribute it and/or modify
it under the terms of the GNU General Public License as published by
the Free Software Foundation, either version 3 of the License, or
(at your option) any later version.\\

If you have any questions or comments
\begin{itemize}
	\item \url{mark@hendryolson.com}
\end{itemize}
\end{frame}

\end{document}